\documentclass[9pt]{article}
\usepackage[letterpaper,left=6.9cm,right=1.9cm,top=3.1cm,bottom=2.3cm]{geometry} % Márgenes y demás
\usepackage[style=apa]{biblatex}
\addbibresource{Bibliografia.bib}


\input{Styles_ACCEFN.sty}  %%Llama el archivo de los estilos


\setlength{\topskip}{0.5cm}
\setlength{\parindent}{0mm}
%\vspace{0.3cm} %Espacio reservado para la revista
{\title{\fontsize{14}{14}\textbf{Tecnologías en la inteligencia artificial para el Marketing: una revisión de la literatura} \\[0.2cm]
\textcolor{gray}{\textbf{Technologies in artificial intelligence
for Marketing: a review of the literature}\\[0.2cm]}}
}

\author[1,*]{\fontsize{9}{9} \textbf{ZAMBRANO NEICER}}

%%%%%% Inicio del documento
\begin{document}

\maketitle
\section*{Resumen}
\justifying
En la actualidad, crear valor agregado y alcanzar nuevas experiencias para el consumidor, potenciadas con inteligencia artificial (IA) se ha convertido en un determinante que puede propiciar ventaja competitiva empresarial. Es por ello que, el objetivo del presente estudio es la revisión de la literatura respecto a las tecnologías de la IA aplicadas al marketing, con el que se beneficiarán tanto la comunidad científica como las empresas en la toma de decisiones estratégicas para publicidad orientada a los gustos del cliente. En este sentido, se describe los principales aportes teóricos, surgimiento, desarrollo, tendencia, perspectivas, componentes, y las contribuciones de las tecnologías de la IA en el marketing. Con este fin, se evaluaron documentos en español e inglés en bases de datos como: Google Académico y Microsoft Academic Search, apoyándose de la herramienta Perish para su búsqueda. Finalmente, se destaca entre otras, la tecnología aprendizaje automático (machine learning) y procesamiento del lenguaje natural (PLN), capaces de potenciar la fidelización de los mercados metas seleccionados.\\[0.2cm]
\textbf{Palabras clave:} Inteligencia artificial, mercadotecnia, empresa, tecnología de la información, aprendizaje automático.


\section*{Abstract}

\justifying
At present, creating added value and reaching new experiences for the consumer, enhanced  with  artificial  intelligence  (AI)  has  become  a  determinant  that  It  can  lead  to competitive business advantage. In this way, the objective of this study is the review of the literature regarding AI technologies applied to marketing, which will benefit both of them the scientific community and enterprise for making strategic decisions to advertising aimed at customer tastes. In this sense, the main theoretical contributions, emergence, development, trend,  perspectives,  components,  and  contributions  of  AI  technologies  in  marketing  are described. To this end, documents in Spanish and English were evaluated in databases such as: Google Scholar and Microsoft Academic Search, using the Perish tool for searching. Finally, it stands out among others, machine learning technology and natural language processing (NLP), capable of enhancing the loyalty of selected target markets.\\[0.2cm]
\textbf{Keywords:} Artificial  intelligence,  marketing,  enterprise,  information  technology,  machine learning.

\newpage
\section*{Introduction}

En la actualidad, el avance vertiginoso de la tecnología ha impactado todos los sectores económicos y, dentro de este contexto, la inteligencia artificial (IA) se ha convertido en un pilar fundamental para el desarrollo empresarial. Como señalan Albarran y Salgado (2013), la inteligencia analítica se ha posicionado como un factor clave para la competitividad empresarial. El procesamiento del lenguaje natural (PLN), por su parte, ha permitido avances significativos en la interacción entre máquinas y humanos, como lo demuestran los estudios de Almeida y Calistru (2013).\cite{albarran2013inteligencia}

A medida que la competencia en el mercado global se intensifica, las empresas deben adaptarse rápidamente a las cambiantes expectativas de los consumidores. La IA facilita esta adaptación al permitir que las organizaciones analicen comportamientos de compra y preferencias con mayor precisión y velocidad que nunca antes. En este sentido, el uso de tecnologías como el aprendizaje automático (machine learning), el procesamiento del lenguaje natural (PLN) y el análisis de big data ha permitido a las empresas obtener una ventaja competitiva significativa. Estos avances no solo mejoran la eficiencia de las campañas de marketing, sino que también potencian la creación de valor añadido para el consumidor, ofreciendo experiencias más relevantes y personalizadas.

El objetivo de este artículo es ofrecer una revisión exhaustiva de la literatura sobre el uso de tecnologías de inteligencia artificial aplicadas al marketing, con el propósito de proporcionar un marco teórico actualizado que apoye tanto a la comunidad científica como a las empresas en la toma de decisiones estratégicas. Al analizar las principales contribuciones teóricas, el desarrollo histórico y las tendencias actuales de la IA en marketing, se busca evidenciar cómo estas tecnologías han transformado el panorama empresarial, permitiendo a las organizaciones no solo satisfacer las necesidades de sus clientes, sino también anticiparse a ellas. 

\section*{Evolución y aportes teóricos de la IA}

La IA ha sido objeto de estudio desde la antigüedad, comenzando con Aristóteles en la antigua Grecia. En 1955, el término "inteligencia artificial" fue acuñado para referirse al desarrollo de máquinas capaces de simular el pensamiento humano. Desde entonces, diversas tecnologías de IA, como las redes neuronales y el aprendizaje automático (machine learning), han evolucionado para transformar áreas como la mercadotecnia. Según Borges-Torres, Arencibia-Ávila y Pérez-Rosell (2018), la toma de decisiones y el enfoque sistémico de la dirección se han visto enriquecidos por las herramientas proporcionadas por la inteligencia artificial.
\cite{borges2018toma}

En marketing, la IA permite la creación de sistemas automatizados que pueden aprender y predecir comportamientos de los consumidores, facilitando una experiencia más personalizada. Estas tecnologías incluyen el procesamiento del lenguaje natural (PLN) y el análisis de emociones, lo que refuerza la relación con el cliente en entornos interactivos.

\section*{Tendencias y perspectivas}

Un ejemplo claro del potencial de la IA en el marketing es el desarrollo de chatbots. Arker (2016) destaca que chatbots como Mitsuku han logrado un alto nivel de interacción humana, lo que demuestra el avance de estas tecnologías en la simulación de conversaciones.
\cite{arker2016chatbot}
\newpage

El procesamiento del lenguaje natural (PLN) y el reconocimiento visual han demostrado ser de gran utilidad en la personalización del servicio al cliente. A través del reconocimiento de texto y la capacidad de procesar datos en tiempo real, las empresas pueden optimizar campañas de marketing y mejorar la experiencia del usuario.

\section*{Impacto en la empresa}

La implementación de la IA en las empresas requiere un marco legal sólido. La Constitución de la República del Ecuador (Asamblea Constituyente, 2008) establece los principios fundamentales para el desarrollo de tecnologías en el país, lo que implica considerar aspectos legales y éticos en la implementación de soluciones basadas en IA.
\cite{asamblea2008constitucion}
\cite{borges2018toma}

La adopción de tecnologías de inteligencia artificial en el ámbito empresarial ha generado una transformación profunda en cómo las empresas gestionan sus operaciones, sus recursos y, sobre todo, la relación con sus clientes. En el caso del marketing, la IA ha permitido la automatización de procesos clave, desde la segmentación del mercado hasta la personalización de campañas publicitarias, lo que ha derivado en un uso más eficiente del tiempo y los recursos. Por ejemplo, herramientas como los sistemas de aprendizaje automático (machine learning) han facilitado la identificación de patrones de comportamiento en los consumidores, lo que permite a las empresas ajustar sus estrategias en tiempo real, respondiendo de manera más ágil y precisa a las demandas del mercado.

El impacto de la IA también se refleja en la capacidad de las empresas para optimizar la experiencia del cliente. Tecnologías como los chatbots y los asistentes virtuales han mejorado significativamente la atención al cliente, permitiendo respuestas inmediatas y soluciones personalizadas. Este tipo de innovación no solo incrementa la satisfacción del cliente, sino que también reduce costos operativos al automatizar tareas repetitivas que antes requerían intervención humana. En consecuencia, las empresas pueden redirigir sus recursos hacia actividades de mayor valor agregado, como la innovación en productos o la mejora continua de sus servicios.

Además, la IA ha revolucionado la forma en que las empresas gestionan sus datos. Con el uso de big data y técnicas avanzadas de análisis predictivo, las organizaciones pueden anticipar comportamientos futuros del mercado, optimizando así sus estrategias de ventas y su capacidad de reacción ante cambios en las tendencias de consumo. Esta ventaja competitiva es especialmente relevante en un entorno globalizado donde las empresas deben ser capaces de adaptarse rápidamente a las fluctuaciones del mercado y las preferencias del consumidor. Así, la inteligencia artificial no solo actúa como una herramienta de optimización, sino como un motor clave para la innovación y la sostenibilidad a largo plazo en el mundo empresarial.

Finalmente, el impacto de la IA en la empresa se extiende más allá de la optimización interna. La capacidad de las tecnologías inteligentes para ofrecer experiencias altamente personalizadas al consumidor también fortalece las relaciones con los clientes, construyendo una lealtad a largo plazo y mejorando la imagen de la marca. En un entorno donde la satisfacción del cliente es clave para la competitividad, las empresas que incorporan IA en su gestión de marketing están mejor posicionadas para liderar el mercado y fomentar un crecimiento sostenible.

\newpage
\section*{Conclusions}

En conclusión, la IA ha demostrado ser una herramienta invaluable para las empresas que buscan mejorar su competitividad y ofrecer mejores experiencias a sus clientes. Si bien Albarran y Salgado (2013) enfatizan la importancia de la inteligencia analítica, Almeida y Calistru (2013) destacan los desafíos asociados con la gestión de grandes volúmenes de datos. Ambos puntos de vista son complementarios y resaltan la complejidad y el dinamismo del campo de la IA en el marketing.
\cite{albarran2013inteligencia}
\cite{almeida2013main}

Sin embargo, a pesar de sus múltiples beneficios, la adopción de la IA aún enfrenta ciertos desafíos. Uno de los principales obstáculos es la falta de conocimiento técnico en algunas áreas empresariales, lo que limita la capacidad de integración de estas tecnologías en las estrategias de marketing. Esto pone de manifiesto la importancia de la formación y capacitación continua en tecnología e inteligencia artificial, para que tanto directivos como equipos de marketing puedan aprovechar al máximo su potencial. Además, el desarrollo de una infraestructura tecnológica adecuada y el manejo ético de los datos también son áreas clave que deben ser abordadas para garantizar el éxito de la implementación de la IA en las empresas.

Otro aspecto que destaca en las conclusiones de esta revisión es que la inteligencia artificial no solo transforma los procesos operativos, sino también la manera en que se concibe la relación empresa-cliente. Al permitir una mayor personalización y precisión en la oferta de productos y servicios, la IA contribuye a una experiencia de cliente más enriquecedora, lo que, a su vez, genera una mayor fidelización y confianza en la marca. Las empresas que han logrado integrar con éxito la IA en sus estrategias de marketing son capaces de establecer relaciones más sólidas y duraderas con sus consumidores, lo que les permite consolidar su posición en el mercado.

A futuro, la inteligencia artificial seguirá evolucionando, incorporando nuevos avances que ampliarán aún más sus aplicaciones en el mundo del marketing. Tecnologías emergentes como el aprendizaje profundo (deep learning), el internet de las cosas (IoT) y la cadena de bloques (blockchain) auguran una mayor automatización, precisión y seguridad en los procesos empresariales, abriendo nuevas oportunidades para innovar en la creación de valor para el cliente. No obstante, el éxito de estas tecnologías dependerá en gran medida de cómo las empresas sean capaces de adaptarse a este nuevo paradigma, invirtiendo en el desarrollo de capacidades tecnológicas y en la creación de una cultura empresarial orientada hacia la innovación.

En conclusión, la inteligencia artificial ha marcado un antes y un después en el marketing, ofreciendo herramientas poderosas que no solo mejoran la eficiencia operativa, sino que también potencian la relación con el cliente. Las organizaciones que apuesten por la integración de estas tecnologías y que sepan adaptarse al cambio estarán mejor posicionadas para liderar en el futuro. Por tanto, es esencial que las empresas sigan explorando nuevas formas de aplicar la IA en sus procesos, para mantenerse competitivas en un mercado cada vez más digital y centrado en el cliente.


\newpage

\printbibliography


\end{document}
